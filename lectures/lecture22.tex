\chapter{Nov.~5 --- Cartan Subalgebras}

\section{Semisimple Elements}

\begin{definition}
  An element $x \in \g$ is called
  \emph{semisimple} if $\ad x \in \End(\g)$
  is a semisimple operator. An
  element $x \in \g$ is called
  \emph{nilpotent} if $\ad x$ is
  nilpotent.
\end{definition}

\begin{theorem}
  If $\g$ is a semisimple complex Lie
  algebra, then any $x \in \g$ can be
  uniquely written as
  \[
    x = x_s + x_n,
  \]
  where $x_s$ is semisimple,
  $x_n$ is nilpotent, and
  $[x_s, x_n] = 0$. Moreover, if for
  some $y \in \g$ we have
  $[x, y] = 0$, then $[x_s, y] = 0$.
\end{theorem}

\begin{proof}
  Uniqueness follows from uniqueness
  in the Jordan decomposition: If
  \[
    x = x_s + x_n = x_s' + x_n',
  \]
  then
  $(\ad x)_s = \ad x_s = \ad x_s'$, so
  $\ad(x_s - x_s') = 0$. Thus
  $x_s = x_s'$ since $\g$ is semisimple.

  For existence, consider
  $\ad x : \g \to \g$, which we can
  decompose into its Jordan blocks, i.e.
  \[
    \g = \bigoplus \g_\lambda,
  \]
  where $(\ad x - \lambda \id)^n|_{\g_\lambda} = 0$
  for $n \gg 0$. Now we use the following
  lemma:
  \begin{quote}
    \vspace{-2em}
    \begin{lemma}
      $[\g_{\lambda}, \g_{\mu}] \subseteq \g_{\lambda + \mu}$.
    \end{lemma}

    \begin{proof}
      By the Jacobi identity, we can write
      \[
        (\ad x - \lambda \id - \mu \id)
        [y, z]
        = [(\ad x - \lambda) y, z]
        + [y, (\ad x - \mu) z]
      \]
      for $y \in \g_\lambda, z \in \g_\mu$.
      Iterating this, we get that
      \[
        (\ad x - \lambda - \mu)^n[y, z]
        = \sum_{k} \binom{n}{k}
        [(\ad x - \lambda)^k y,
        (\ad x - \mu)^{n - k} z].
      \]
      This vanishes for sufficiently
      large $n$, so we get the result.
    \end{proof}
  \end{quote}
  Write
  $\ad x = (\ad x)_s + (\ad x)_n$, so
  $(\ad x)_s|_{\g_\lambda} = \lambda$. The
  lemma implies that
  $(\ad x)_s$ is a derivation. Then
  it must be an inner derivation, so
  $(\ad x)_s = \ad x_s$ for some
  $x_s \in \g$.
\end{proof}

\begin{corollary}
  In any semisimple Lie algebra, there
  exists nonzero semisimple elements.
\end{corollary}

\begin{proof}
  If there are no nontrivial semisimple
  elements, then $x = x_n$ for every
  $x \in \g$. Then Engel's theorem,
  $\g$ is nilpotent, contradicting
  semisimplicity.
\end{proof}

\begin{definition}
  A Lie subalgebra
  $\mathfrak{h} \subseteq \g$ is called
  \emph{toral} if it is commutative and
  consists of semisimple elements.
\end{definition}

\begin{theorem}
  Let $\g$ be a complex semisimple Lie
  algebra and $\mathfrak{h}$ a toral
  subalgebra. Let $(\cdot, \cdot)$ be a
  non-degenerate invariant bilinear
  form on $\g$ (e.g. the Killing form).
  Then
  \begin{enumerate}
    \item $\g = \bigoplus_{\alpha \in \mathfrak{h}^*} \g_\alpha$,
      where $\g_\alpha$ is a common
      eigenspace for all operators
      $\ad_h$, with $h \in \mathfrak{h}$,
      with eigenvalue $\alpha$, i.e.
      we have
      \[
        \ad_h x = \langle \alpha, h \rangle x, \quad x \in \g_\alpha.
      \]
      In particular, $\mathfrak{h} \subseteq \g_0$.
    \item $[\g_\alpha, \g_\beta] \subseteq \g_{\alpha + \beta}$.
    \item If $\alpha + \beta \ne 0$, then
      $\g_\alpha, \g_\beta$ are orthogonal
      with respect to $(\cdot, \cdot)$.
    \item For every $\alpha$, the
      form $(\cdot, \cdot)$ gives a
      non-degenerate pairing
      $\g_{\alpha} \otimes \g_{-\alpha} \to \C$.
  \end{enumerate}
\end{theorem}

\begin{proof}
  (1) This follows directly from our
  previous results.

  (2) Let $y \in \g_\alpha$ and
  $z \in \g_\beta$. We can write
  \[
    \ad_h [y, z]
    = [\ad_h y, z] + [y, \ad_h z]
    = [\langle h, \alpha \rangle y, z]
    + [y, \langle h, \beta \rangle z]
    = \langle h, \alpha + \beta \rangle [y, z],
  \]
  which proves the claim.

  (3) Let $x \in \g_\alpha$
  and $y \in \g_\beta$.
  For any $h \in \mathfrak{h}$,
  by the
  invariance of $(\cdot, \cdot)$ we have
  \[
    0 = ([h, x], y) + (x, [h, y])
    = (\langle h, \alpha \rangle x, y)
    + (x, \langle h, \beta \rangle y)
    = (\langle h, \alpha \rangle + \langle h, \beta \rangle)
    (x, y),
  \]
  which gives $(x, y) = 0$ since
  $\alpha + \beta \ne 0$.
\end{proof}

\begin{lemma}
  We have the following:
  \begin{enumerate}
    \item The restriction of
      $(\cdot, \cdot)$ to $\g_0$ is
      non-degenerate.
    \item If $x \in \g_0$ with
      $x = x_s + x_n$, then
      $x_s, x_n \in \g_0$.
    \item $\g_0$ is a reductive
      subalgebra of $\g$.
  \end{enumerate}
\end{lemma}

\section{Cartan Subalgebras}

\begin{definition}
  Let $\g$ be a complex semisimple
  Lie algebra. A \emph{Cartan subalgebra}
  $\mathfrak{h} \subseteq \g$ is a toral
  subalgebra which coincides with
  its centralizer
  $C(\mathfrak{h}) = \{x : [x, h] = 0 \text{ for all } h \in \mathfrak{h}\}$.
\end{definition}

\begin{example}
  Let $\g = \mathfrak{sl}(n, \C)$. Then a
  Cartan subalgebra is
  \[
    \mathfrak{h} =
    \{\text{diagonal matrices with trace $0$}\}.
  \]
  If we take $h \in \mathfrak{h}$ such
  with all eigenvalues distinct, then
  $[x, h] = 0$ implies
  \[
    [h, E_{i, j}]
    = (h_i - h_j) E_{i, j}
  \]
  where $h = \diag(h_1, \ldots, h_n)$, 
  so $x$ is also diagonal. Thus
  $C(\mathfrak{h}) = \mathfrak{h}$.
\end{example}

\begin{theorem}
  Let $\mathfrak{h} \subseteq \g$ be a
  maximal toral subalgebra. Then
  $\mathfrak{h}$ is a Cartan subalgebra.
\end{theorem}

\begin{proof}
  Let $\g = \bigoplus \g_\alpha$ for
  $\ad_\mathfrak{h}$. We first show that
  $\g_0 = C(\mathfrak{h})$ is toral, which will
  imply that $C(\mathfrak{h}) = \mathfrak{h}$.
  Note that for $x \in \g_0$,
  we have ${\ad_x}|_{\g_0}$ is nilpotent
  (otherwise, $x_s$ has a nonzero eigenvalue
  and $x_s \in \g_0$ is a semisimple
  element, so ${\ad_{x_s}}|_{\g_0} \ne 0$
  implies $x_s \notin \mathfrak{h}$).
  By Engel's theorem, $\g_0$ is
  nilpotent, so $\g_0$ is
  reductive, so $\g_0$ is commutative.

  Now we show that any element
  $x \in \g_0$ is semisimple. We show that
  $x_n = 0$ for every $x$. Note that
  $\ad_{x_n}$ is nilpotent and
  $\g_0$ is commutative, so for
  any $y \in \g_0$, we have
  $\ad_{x_n} \ad_y$ is also nilpotent.
  Then
  \[
    \tr_{\g_0}(\ad_{x_n} \ad_y) = 0.
  \]
  Since the Killing form is
  $0$, we must have $\ad_{x_n} = 0$, so
  $x_n = 0$.
\end{proof}

\begin{corollary}
  For any complex semisimple Lie
  algebra $\g$, there exists a
  Cartan subalgebra.
\end{corollary}

\begin{definition}
  Define the \emph{rank} of $\g$
  to be $\rank(\g) = \dim \mathfrak{h}$.
\end{definition}

\begin{example}
  $\rank(\mathfrak{sl}(n, \C)) = n - 1$.
\end{example}

\section{Root Decomposition}

\begin{theorem}
  Let $\g$ be a semisimple Lie algebra.
  \begin{enumerate}
    \item We have the following decomposition
      for $\g$
      (called the \emph{root decomposition}):
      \[
        \g = \mathfrak{h} \oplus
        \bigoplus_{\alpha \in R} \g_\alpha,
      \]
      where $\g_\alpha = \{x : [h, x] = \langle \alpha, h \rangle x \text{ for all } x \in \mathfrak{h}\}$
      and $R = \{\alpha \in \mathfrak{h}^* \setminus \{0\} : \g_\alpha \ne \{0\}\}$.
      We call $R$ a \emph{root system} of
      $\g$ and $\{\g_\alpha\}$
      \emph{root subspaces.}
    \item $[\g_\alpha, \g_\beta] \subseteq \g_{\alpha + \beta}$,
      where $\g_0 = \mathfrak{h}$.
    \item If $\alpha + \beta \ne 0$,
      then $\g_\alpha, \g_\beta$ are
      orthogonal with respect to the
      Killing form.
    \item For every $\alpha$, the
      Killing form gives a non-degenerate
      pairing $\g_\alpha \otimes \g_{-\alpha} \to \C$.
      In particular, the restriction of
      $K$ to $\mathfrak{h}$ is
      non-degenerate.
  \end{enumerate}
\end{theorem}

\begin{theorem}
  Let $\g_1, \dots, \g_n$ be simple
  Lie algebras and $\g = \bigoplus \g_i$. Then
  \begin{enumerate}
    \item If $\mathfrak{h}_i \subseteq \g_i$
      is a Cartan subalgebra of $\g_i$
      and $R_i \subseteq \mathfrak{h}_i^*$
      is a root system, then
      $\mathfrak{h} = \bigoplus \mathfrak{h}_i$
      is a Cartan subalgebra of $\g$ 
      and $R = \bigsqcup_i R_i$ is a
      root system of $\g$.
    \item Every Cartan subalgebra in $\g$
      has a decomposition
      $\mathfrak{h} = \bigoplus \mathfrak{h}_i$
      as above.
  \end{enumerate}
\end{theorem}

\begin{example}
  Let $\g = \mathfrak{sl}(n, \C)$
  and $\mathfrak{h} = \{\text{diagonal matrices with trace $0$}\}$.
  Define linear functionals
  \[
    e_i =
    \begin{pmatrix}
      h_1 & 0 & \cdots & 0 \\
      0 & h_2 & \cdots & 0 \\
      \vdots & \vdots & \ddots & \vdots \\
      0 & 0 & \cdots & h_n
    \end{pmatrix}
    \longmapsto h_i.
  \]
  Note that $\sum e_i = 0$. We can write
  $\mathfrak{h}^* = (\bigoplus_{i = 1}^n \C e_i) / \C(e_1 + \dots + e_n)$.
  Then $E_{i, j}$ are eigenvectors
  for $\ad_h$ for every $h \in \mathfrak{h}$,
  since
  \[
    [h, E_{i, j}]
    = (h_i - h_j) E_{i, j}
    = (e_i - e_j)(h) E_{i, j}.
  \]
  Thus the root system is
  \[
    R = \{e_i - e_j : i \ne j\}
    \subseteq \frac{\bigoplus_{i = 1}^n \C e_i}
    {\C(e_1 + \dots + e_n)},
  \]
  with root subspaces
  $\g_{e_i - e_j} = \C E_{i, j}$. The
  Killing form on $\mathfrak{h}$ is given by
  \[
    (h, h')
    = \sum_{i \ne j} (h_i - h_j) (h_i' - h_j')
    = 2n \sum h_i h_i'
    = 2n \tr(h h').
  \]
  Now we find the
  dual pairing on $\mathfrak{h}^*$:
  If $\lambda = \sum \lambda_i e_i \in \mathfrak{h}^*$, $\mu = \sum \mu_i e_i \in \mathfrak{h}^*$
  with $\sum_i \lambda_i = \sum_i \mu_i = 0$, then
  \[
    (\lambda, \mu)
    = \frac{1}{2n} \sum_i \lambda_i \mu_i.
  \]
  We have a natural isomorphism
  $\mathfrak{h} \to \mathfrak{h}^*$ given by
  $(\cdot, \cdot)$, which is defined by
  \[
    (\alpha, \beta)
    = \langle H_\alpha, \beta \rangle
    = (H_\alpha, H_\beta)
  \]
  for $\alpha, \beta \in \mathfrak{h}^*$,
  where $H_\alpha, H_\beta \in \mathfrak{h}$
  are the corresponding elements.
\end{example}

\begin{lemma}\label{lem:root-commutator}
  Let $e \in \g_\alpha$ and
  $f \in \g_{-\alpha}$. Then
  $[e, f] = (e, f) H_\alpha$.
\end{lemma}

\begin{proof}
  By the invariance of $(\cdot, \cdot)$, for any $h \in \mathfrak{h}$
  we have
  \[
    ([e, f], h)
    = (e, [f, h])
    = -(e, [h, f])
    = \langle f, \alpha \rangle (e, f)
    = (e, f) (h, H_\alpha),
  \]
  which proves the desired claim.
\end{proof}
