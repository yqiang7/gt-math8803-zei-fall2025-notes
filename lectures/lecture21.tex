\chapter{Nov.~3 --- Semisimple Lie Algebras, Part 3}

\section{Properties of Semisimple Lie Algebras, Continued}
\begin{prop}
  Let $\g = \g_1 \oplus \cdots \oplus \g_k$
  with $\g_i$ simple. Then any ideal
  $I$ in $\g$ has the form
  \[
    I = \bigoplus_{i \in J} \g_i,
    \quad J \subseteq \{1, \dots, k\}.
  \]
\end{prop}

\begin{proof}
  The proof is by induction on $k$.
  Let $\pi_k : \g \to \g_k$, so
  $\pi_k(I) \subseteq \g_k$.
  Since $\g_k$ is simple, we
  either have $\pi_k(I) = 0$ or
  $\pi_k(I) = \g_k$. In the first case,
  $I \subseteq \g_1 \oplus \cdots \oplus \g_{k-1}$,
  so we are done by the induction hypothesis.
  Otherwise if $\pi_k(I) = \g_k$, we have
  \[
    [\g_k, I]
    = [\g_k, \pi_k(I)]
    = \g_k.
  \]
  Since $I$ is an ideal, this implies
  $\g_k \subseteq I$, so
  $I = I' \oplus \g_k$ with
  $I' \subseteq \g_1 \oplus \cdots \oplus \g_{k-1}$,
  and again we are done by the induction hypothesis.
\end{proof}

\begin{corollary}
  Any ideal in a semisimple Lie algebra
  is semisimple. Also, any quotient
  of a semisimple Lie algebra is semisimple.
\end{corollary}

\begin{prop}
  If $\g$ is a semisimple Lie algebra
  and $G$ is the connected Lie group
  corresponding to $\g$, then
  $\Der \g = \g$ and
  $\Aut(\g) / {\Ad G}$ is discrete,
  where $\Ad G = G / \mathcal{Z}(G)$.
\end{prop}

\begin{proof}
  For any $x \in \g$, we have
  an inner derivation $\ad_x$. The
  natural morphism $\g \to \Der \g$
  is injective since $\mathfrak{z}(\g) = 0$.
  Now let $\delta : \g \to \g$ be a
  derivation, so
  \[
    \delta[a, b]
    = [\delta a, b] + [a, \delta b].
  \]
  Then $\ad_{\delta(x)} = [\delta, \ad_x] = \delta \ad_x - \ad_x \delta$,
  where $\delta, \ad_x$ are operators
  from $\End(\g)$. So we can view
  $\g \subseteq \Der(\g)$ as an ideal.
  Extend the Killing form to $\Der(\g)$
  by setting
  \[
    K(\delta_1, \delta_2)
    = \tr_\g(\delta_1 \delta_2).
  \]
  Let $I = \g^\perp \subseteq \Der(\g)$,
  which is an ideal since $K$ is invariant.
  The restriction of $K$ to $\g$ is
  non-degenerate, so $I \cap \g = 0$.
  Thus $\Der \g = I \oplus \g$. These
  are ideals, so we have
  $[I, \g] = 0$. Since
  \[
    \ad_{\delta(x)}
    = [\delta, \ad_x] = 0
  \]
  for all $\delta \in I$, we have
  $\delta(x) = 0$, so $I = 0$.
\end{proof}

\begin{example}
  Let $\g = \mathfrak{u}(n)$ and
  consider the trace form
  $(x, y) = \tr(xy)$.
  Note $\tr(xy) = -\tr(x \overline{y}^T)$, so
  \[
    \tr(x, x)
    = -\sum_{i, j} |x_{i, j}|^2.
  \]
  In particular, we see that
  the trace form is negative definite.
\end{example}

\begin{theorem}
  Let $G$ be a compact real Lie group and
  $\g = \Lie(G)$ (which is reductive).
  Then the Killing form of $\g$ is
  negative semidefinite with
  $\ker K = \mathfrak{z}(\g)$. Moreover,
  the Killing form of the semisimple
  part $\g / \mathfrak{z}(\g)$
  is negative definite. Conversely,
  if $\g$ is a semisimple real Lie algebra
  with negative definite Killing form, then
  $\g$ is the Lie algebra of some compact
  real Lie group.
\end{theorem}

\begin{proof}
  We have proved that when $G$ is
  compact, any complex representation
  is unitary. So we have
  $\rho : G \to \GL(V)$
  where $\rho(G) \subseteq U(V)$.
  Then we have
  $\rho(\g) \subseteq \mathfrak{u}(V)$,
  so the trace form $B_V(x, y)$
  is negative semidefinite, and
  $\ker B_V = \ker \rho$. Applying
  this to the complexified adjoint
  representation
  $V = \g_\C$ implies that
  the Killing form is negative
  semidefinite and $\ker K = \mathfrak{z}(\g_\C)$.

  Now assume $\g$ is a real Lie
  algebra and $K$ is negative definite.
  Let $G$ be the connected Lie group
  for $\g$, and define $B(x, y) = -K(x, y)$,
  which is positive definite and leaves
  $\Ad(G)$ invariant. By
  choosing an appropriate basis, we can
  embed $\Ad(G) \subseteq \SO(\g)$.
  We know that $\Ad(G)$ is the connected
  component of the identity in
  $\Aut(\g)$, and
  $\Aut(\g) \subseteq \GL(\g)$ is a
  closed Lie subgroup, so
  $\Ad(G)$ is a closed Lie subgroup
  of $\SO(\g)$. Thus $\Ad(G)$ is
  compact.
\end{proof}

\begin{remark}
  One can actually prove a stronger result:
  Let $\g$ be a real Lie algebra with
  negative definite Killing form.
  Then any connected Lie group with
  Lie algebra $\g$ is compact. In particular,
  the simply-connected one is compact.
\end{remark}

\begin{exercise}
  If $\g$ is a real Lie algebra with
  positive definite Killing form, then
  show that $\g = 0$.
\end{exercise}

\begin{theorem}
  Let $\g$ be a complex semisimple Lie
  algebra. Then there exists a real
  subalgebra $\mathfrak{k}$ such that
  $\mathfrak{k} \otimes \C = \g$ and
  $\mathfrak{k}$ is the Lie algebra of a
  compact Lie group $K$. This $\mathfrak{k}$
  is known as a compact form of
  $\g$, and it is unique up to
  conjugation. If $G$ is a complex Lie
  group corresponding to $\g$, then
  one can choose $K \subseteq G$
  (known as a compact form of $G$).
\end{theorem}

\begin{example}
  For $\g = \mathfrak{sl}(n, \C)$,
  we have $\mathfrak{k} = \mathfrak{su}(n)$.
\end{example}

\section{The Casimir Element}

\begin{prop}
  Let $\g$ be a Lie algebra and
  $B$ a non-degenerate invariant symmetric
  bilinear form on $\g$. Let
  $\{x_i\}_{i = 1}^{\dim \g}$ be a basis
  of $\g$ and
  $\{x^i\}_{i = 1}^{\dim \g}$ the dual basis
  with respect to $B$. Then
  \[
    C_B = \sum_i x_i x^i \in U(\g)
  \]
  does not depend on the choice of basis and
  is central in $U(\g)$. It is called
  the \emph{Casimir element} determined by
  $B$. When $\g$ is semisimple, $C_K$ is
  called the \emph{Casimir element} of $\g$.
\end{prop}

\begin{proof}
  Define $I = \sum_{i = 1}^{\dim \g} (x_i \otimes x^i) \in \g \otimes \g$.
  As an element of $\End(\g)$, it is the
  identity operator.
  Thus $I$ is $\ad_\g$-invariant.
  Since the multiplication
  map $\g \otimes \g \to U(\g)$ is a
  morphism of representations, we get that
  $C_B = \sum x_i x^i$ is $\ad_\g$-invariant,
  so it is central.
\end{proof}

\begin{example}
  For $\g = \mathfrak{sl}(2, \C)$,
  we have $C = \frac{1}{2} h^2 + ef + fe$.
\end{example}

\begin{remark}
  For simple Lie algebras, the invariant
  bilinear form is unique up to a constant.
  In particular, the Casimir element is
  unique up to a constant in
  this case.

  To see this, let
  $B_1(x, y)$ and $B_2(x, y)$ be two
  bilinear forms on a simple Lie algebra
  $\g$. Then they differ by an operator
  $A : \g \to \g$ which is
  $\ad_\g$-invariant. Since $\g$ is simple,
  Schur's lemma implies $A$ is a constant.
\end{remark}

\begin{prop}
  Let $V$ be a non-trivial irreducible
  representation of a semisimple
  Lie algebra $\g$. Then there exists a
  central element $C_V \in \mathcal{Z}(U(\g))$
  which acts by a non-zero constant on $V$
  and which acts by $0$ on the
  trivial representation.
\end{prop}

\begin{proof}
  Let $B_V(x, y) = \tr(\rho_V(x) \rho_V(y))$.
  If $B_V$ is non-degenerate, then we
  can take $C_V = C_{B_V}$. In this
  case, by
  Schur's lemma we obtain the result since
  \[
    \tr(C_V)
    = \sum \tr(x_i x^i)
    = \dim V,
  \]
  so $C_V = \lambda \id$ implies that
  $\lambda \ne 0$. In general,
  let $I = \ker B_V \subseteq \g$. Then
  $I$ is an ideal in $\g$ and
  $I \ne \g$ (otherwise
  $\rho(\g) \subseteq \gl(V)$
  would be solvable,
  which is not possible). Thus
  $\g = I \oplus \g'$ for some ideal
  $\g'$ which is semisimple, so
  $B_V|_{\g'}$ is non-degenerate and
  we can repeat the above procedure.
\end{proof}

\begin{remark}
  One can prove the following stronger
  result: The Casimir element
  defined by the Killing form acts by a
  nonzero constant in any non-trivial
  representation.
\end{remark}
