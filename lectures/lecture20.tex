\chapter{Oct.~29 --- Semisimple Lie Algebras, Part 2}

\section{Invariant Bilinear Forms}

\begin{definition}
  Recall that a bilinear form $B$
  on $\g$ is \emph{invariant} if
  \[
    B(\ad_x y, z) + B(y, \ad_x z) = 0.
  \]
\end{definition}

\begin{lemma}
  Let $B$ be an invariant bilinear
  form on $\g$
  and $I \subseteq \g$ an ideal. Then
  \[
    I^\perp
    = \{x \in \g : B(x, y) = 0 \text{ for all } y \in \g\}
  \]
  is also an ideal in $\g$. In particular,
  $\g^\perp = \ker B$ is an ideal.
\end{lemma}

\begin{remark}
  In general,
  $\g \ne I \oplus I^\perp$ as we may
  have $I \cap I^\perp \ne 0$.
\end{remark}

\begin{example}
  Let $\g = \gl(n, \C)$
  and $B(x, y) = \tr(xy)$. Then
  \[
    \tr([x, y], z)
    + \tr(y, [x, z])
    = \tr(xyz - yxz + yxz - yzx)
    = \tr([x, yz]) = 0,
  \]
  so $B$ is invariant. An even easier
  proof of this is $\tr(gyg^{-1} gzg^{-1}) = \tr(yz)$
  for any $g \in \GL(n, \C)$, so
  $B$ is invariant under the adjoint
  action of $\GL(n, \C)$, hence it is
  also invariant under the action by $\gl(n, \C)$.
\end{example}

\begin{prop}
  Let $V$ be a representation
  of $\g$ with $\rho : \g \to \gl(V)$, and
  define
  \[
    B_V(x, y) = \tr_V(\rho(x) \rho(y)).
  \]
  Then $B$ is a symmetric invariant
  bilinear form on $\g$.
\end{prop}

\begin{theorem}
  Let $\g$ be a Lie algebra with
  a representation $V$ such that
  the form $B_V$ is non-degenerate.
  Then $\g$ is reductive.
\end{theorem}

\begin{proof}
  We need to show that
  $[\g, \rad(\g)] = 0$.
  Let $x \in [\g, \rad(\g)]$, so
  $x$ acts as $0$ in any irreducible
  representation. Thus
  $x \in \ker B_V$ for any
  irreducible $V$. Now if we have a
  short exact sequence
  \begin{center}
    \begin{tikzcd}
      0 \ar[r] & V_1 \ar[r] & W \ar[r] & V_2 \ar[r] & 0
    \end{tikzcd}
  \end{center}
  then $B_W = B_{V_1} + B_{V_2}$, so
  $x \in \ker B_V$ for all $V$.
  Thus $x = 0$, since $B_V$ is
  non-degenerate for some $V$.
\end{proof}

\begin{theorem}
  All classical Lie algebras are
  reductive. Moreover,
  $\mathfrak{sl}(n, \K)$, $\mathfrak{so}(n, \K)$ (for $n > 2$),
  $\mathfrak{su}(n)$,
  $\mathfrak{sp}(n, \K)$ are semisimple;
  $\gl(n, \K)$, $\mathfrak{u}(n)$
  have 1-dimensional center, and
  $\gl(n, \K) = {\K \id} \oplus {\mathfrak{sl}(n, \K)}$,
  $\mathfrak{u}(n) = {i \R \id} \oplus {\mathfrak{su}(n)}$.
\end{theorem}

\begin{proof}
  Take $B_V$ with $V$ being the
  definition representation, i.e.
  \[
    V =
    \begin{cases}
      \K^n & \text{for all except $\mathfrak{sp}(n, \K)$}, \\
      \K^{2n} & \text{for $\mathfrak{sp}(n, \K)$}.
    \end{cases}
  \]
  Then we can check that:
  \begin{itemize}
    \item for $\gl(n, \K)$, we have
      $B(x, y) = \tr(x, y) = \sum_{i, j} x_{i, j} y_{j, i}$,
      which is non-degenerate;
    \item for $\mathfrak{sl}(n, \K)$,
      it follows from the
      $\gl(n, \K)$ case and
      $\gl(n, \K) = {\K \id} \oplus {\mathfrak{sl}(n, \K)}$;
    \item for $\mathfrak{so}(n, \K)$,
      we have $B(x, y) = \sum x_{i, j} y_{j, i} = - 2\sum_{i < j} x_{i, j} y_{i, j}$,
      which is non-degenerate;
    \item for $\mathfrak{u}(n)$, we have
      $B(x, y) = - \tr(x \overline{y}^t) = - \sum_{i < j} x_{i, j} \overline{y}_{i, j}$,
      which is non-degenerate.
    \item for $\mathfrak{su}(n)$, it follows
      from the result for $\mathfrak{u}(n)$
      since $\mathfrak{su}(n) \subseteq \mathfrak{u}(n)$.
  \end{itemize}
  Note that the bilinear form
  for $\mathfrak{u}(n, \K)$
  is negative-definite.
  The case $\mathfrak{sp}(n, \K)$
  is left as an exercise.
\end{proof}

\section{Cartan's Criterion}

\begin{definition}
  The \emph{Killing form} on $\g$
  is the bilinear form
  \[
    K(x, y) = \tr_\g(\ad_x \ad_y).
  \]
\end{definition}

\begin{remark}
  Let $\mathfrak{h}$ be a subalgebra in $\g$.
  We write $K^{\mathfrak{h}}$ for
  the Killing form with
  taking the trace over $\mathfrak{h}$:
  \begin{align*}
    K^{\mathfrak{h}}
    : \mathfrak{h} \otimes \mathfrak{h}
    &\longrightarrow \K \\
    (x, y)
    &\longmapsto \tr_{\End \mathfrak{h}}(\ad_x \ad_y),
  \end{align*}
  and $K^\g$ for the
  Killing form with
  taking the trace over $\g$:
  \begin{align*}
    K^{\g}
    : \mathfrak{h} \otimes \mathfrak{h}
    &\longrightarrow \K \\
    (x, y)
    &\longmapsto \tr_{\End \g}(\ad_x \ad_y).
  \end{align*}
\end{remark}

\begin{exercise}
  Show that if $I \subseteq \g$ is an ideal,
  then $K^I$ coincides with
  $K^\g : I \times I \to \K$.
\end{exercise}

\begin{example}
  Let $\g = \mathfrak{sl}(2, \C)$ with
  basis $\{e, h, f\}$. We have
  \[
    \ad_e =
    \begin{pmatrix}
      0 & -2 & 0 \\
      0 & 0 & 1 \\
      0 & 0 & 0
    \end{pmatrix}, \quad
    \ad_h =
    \begin{pmatrix}
      2 & 0 & 0 \\
      0 & 0 & 0 \\
      0 & 0 & -2
    \end{pmatrix}, \quad
    \ad_f =
    \begin{pmatrix}
      0 & 0 & 0 \\
      -1 & 0 & 0 \\
      0 & 2 & 0
    \end{pmatrix}.
  \]
  Then $K(h, h) = 8$ and
  $K(e, f) = K(f, e) = 4$.
\end{example}

\begin{theorem}[Cartan's criterion for solvability]\label{thm:cartan-solvable}
  A Lie algebra $\g$ is solvable
  if and only if
  \[K([\g, \g], \g) = 0.\]
\end{theorem}

\begin{theorem}[Cartan's criterion for semisimplicity]\label{thm:cartan-semisimple}
  A Lie algebra is semisimple
  if and only if its Killing form
  is non-degenerate.
\end{theorem}

\begin{theorem}\label{thm:decomposition}
  Let $V$ be a finite-dimensional
  vector space.
  \begin{enumerate}
    \item Any linear operator
      $A$ can be uniquely written as a sum
      \[
        A = A_s + A_n,
      \]
      where $A_s A_n = A_n A_s$ and
      $A_s$ is semisimple, $A_n$ is
      nilpotent.
    \item For any linear operator
      $A : V \to V$, define
      $\ad_A : \End V \to \End V$ by
      $\ad_A B = AB - BA$. Then
      \[
        (\ad_A)_s = \ad A_s,
      \]
      and $\ad A_s = P(\ad_A)$ for
      some polynomial
      $P \in t \C[t]$.\footnote{Note that $0$ is always an eigenvalue as $\ad_A A = 0$.}
    \item Define $\overline{A}_s$ to be the
      operators with complex conjugate
      eigenvalues as $A_s$, i.e.
      $A_s v = \lambda v$ if and only if
      $\overline{A}_s v = \overline{\lambda} v$.
      Then $\ad \overline{A}_s$
      can be written in the form
      $\ad \overline{A}_s = Q(\ad_A)$
      for some $Q \in t \C[t]$.
  \end{enumerate}
\end{theorem}

\begin{proof}[Proof of Theorem \ref{thm:cartan-solvable}]
  If $\g$ is real, then $\g$ is solvable
  if and only if $\g_\C$ is solvable,
  and $K([\g, \g], \g) = 0$
  if and only if $K([\g_\C, \g_\C], \g_\C) = 0$.
  So we may assume that $\g$ is complex.

  $(\Rightarrow)$ If $\g$ is
  solvable, then by Lie's theorem
  there exists a basis such that all
  $\ad_x$ are upper-triangular. Then
  $[\g, \g]$ is strictly upper-triangular,
  so $K([\g, \g], \g) = 0$.

  $(\Leftarrow)$ We use the following
  lemma:
  \begin{quote}
    \vspace{-2em}
    \begin{lemma}
      Let $V$ be a complex vector space
      and $\g \subseteq \gl(V)$ a Lie
      subalgebra such that for all
      $x \in [\g, \g]$ and
      $y \in \g$, we have
      $\tr(xy) = 0$. Then
      $\g$ is solvable.
    \end{lemma}

    \begin{proof}
      Since $x \in [\g, \g]$, we can
      write $x = x_s + x_n$ by
      Theorem \ref{thm:decomposition}.
      Then
      \[
        \tr(x \overline{x}_s)
        = \sum \lambda_i \overline{\lambda}_i
        = \sum |\lambda_i|^2,
        \tag{$*$}
      \]
      where the $\lambda_i$ are the
      eigenvalues of $x$. Now
      $x \in [\g, \g]$ implies
      that $x = \sum_i [y_i, z_i]$, so
      \[
        \tr(x \overline{x}_s)
        = \tr\left(\sum_i [y_i, z_i] \overline{x}_s\right)
        = - \sum_i \tr(y_i [\overline{x}_s, z_i])
        = 0,
      \]
      where $[\overline{x}_s, z_i] = \ad_{\overline{x}_s} z_i = Q(\ad_{x}) z_i \in [\g, \g]$.
      Thus $x$ is nilpotent by $(*)$
      (all of its eigenvalues are zero), which
      implies $[\g, \g]$ is nilpotent,
      so $\g$ is solvable.
    \end{proof}
  \end{quote}
  Thus if $K(\g, [\g, \g]) = 0$,
  then by the lemma $\ad \g \subseteq \gl(\g)$
  is solvable. So both $\mathfrak{z}(\g)$
  and $\g / \mathfrak{z}(\g) = \ad \g$
  are solvable, hence $\g$ is solvable.
\end{proof}

\begin{proof}[Proof of Theorem \ref{thm:cartan-semisimple}]
  $(\Leftarrow)$ If
  $x \in \mathfrak{z}(\g)$, then
  $\ad_x = 0$, so
  $x \in \ker(K)$. Thus
  $\mathfrak{z}(\g) = 0$, so
  $\g$ is semisimple.

  $(\Rightarrow)$ If
  $\g$ is semisimple, then
  take $I = \ker K$. Note that
  $I$ is an ideal in $\g$, and
  the restriction of $K$ to $I$ coincides
  with $K^I$, so $K^I$ is trivial
  on $I$. Hence $I$ is solvable, so
  $I = 0$ since $\g$ is semisimple.
\end{proof}

\begin{theorem}
  Let $\g$ be a semisimple
  Lie algebra. If $I \subseteq \g$
  is an ideal, then there exists an ideal
  $I'$ such that $\g = I \oplus I'$.
\end{theorem}

\begin{proof}
  Let $I^\perp$ be the orthogonal complement
  of $I$ with respect to $K$, which is
  also an ideal. Then
  $I \cap I^\perp$ is an ideal
  in $\g$ with zero Killing form, so
  $I \cap I^\perp$ is solvable, which
  implies $I \cap I^\perp = \{0\}$.
\end{proof}

\begin{corollary}
  A Lie algebra is semisimple if and only
  if it is a direct sum of simple
  Lie algebras.
\end{corollary}

\pagebreak
\begin{corollary}
  If $\g$ is semisimple, then
  $[\g, \g] = \g$.
\end{corollary}

\begin{prop}\label{prop:real-complex-semisimple}
  Let $\g$ be a real Lie algebra and
  $\g_\C$ its complexification. Then
  $\g$ is semisimple if and only if
  $\g_\C$ is semisimple.
\end{prop}

\begin{remark}
  Proposition
  \ref{prop:real-complex-semisimple}
  does not hold with
  ``semisimple'' replaced by ``simple'':
  We have
  \[\mathfrak{so}(3, 1)_\C = \mathfrak{sl}(2, \C) \oplus \mathfrak{sl}(2, \C),\]
  but $\mathfrak{so}(3, 1)$ is simple.
\end{remark}
