\chapter{Nov.~17 --- Weyl Chambers}

\section{Weyl Chambers}

\begin{definition}
  Recall that the Weyl group $W$
  is generated the reflections
  $s_\alpha$ across the
  hyperplane $L_\alpha$
  (i.e. $\alpha^\perp$). A
  \emph{Weyl chambers} $C$ is a
  connected component of
  $E \setminus \bigcup_{\alpha \in R} L_\alpha$.
\end{definition}

\begin{remark}
  There is a one-to-one
  correspondence between
  $R_+$ (the set of positive roots),
  $\Pi$ (the set of simple roots),
  and Weyl chambers: Given a
  Weyl chamber $C_+$, we can define
  a set of positive roots
  \[
    R_+ = \{
      \alpha \in R : (\alpha, t) > 0
      \text{ for all } t \in C_+
    \}.
  \]
  Conversely, given a
  set $R_+$ of positive roots,
  we can define a Weyl chamber
  \[
    C_+ =
    \{
      t \in E : (\alpha, t) > 0
      \text{ for all } \alpha \in R_+
    \}
    = \{
      t \in E : (\alpha, t) > 0
      \text{ for all } \alpha \in \Pi
    \}.
  \]
\end{remark}

\begin{prop}
  Let $C$ be a Weyl chamber. Then
  \begin{enumerate}
    \item $C$ is an unbounded
      convex cone.
    \item $\overline{C} \setminus C$
      is a union of codimension
      $1$ faces. These faces
      $F_i \subseteq L_\alpha$
      are called the
      \emph{walls} of $C$, and
      $\alpha \in \Pi$.
  \end{enumerate}
\end{prop}

\begin{theorem}
  The action of the Weyl group
  $W$ on the Weyl chambers $\{C\}$
  is transitive, i.e. if
  $C, C'$ are Weyl chambers, there
  there exists $w \in W$ such that
  $w(C) = C'$.
\end{theorem}

\begin{proof}
  Call a Weyl chamber
  $C_1$ \emph{adjacent} to $C_2$
  if they have a codimension $1$
  face in common. Then:
  \begin{enumerate}
    \item Any two Weyl chambers
      $C, C'$
      are connected by a chain of
      adjacent Weyl chambers.
    \item If $C_1$ is adjacent to
      $C_2$ via the common
      codimension $1$ face
      $\subseteq L_\alpha$, then
      $C_1 = s_\alpha(C_2)$.
  \end{enumerate}
  (One can pick a line going through
  $C, C'$ not going through the
  origin, then the Weyl chambers
  which this line intersects form a
  chain which connects $C$ and $C'$.)
  So let
  \[
    C = C_0, C_1, \ldots, C_k = C'
  \]
  be the chain of adjacent Weyl
  chambers from $C$ to $C'$.
  Let $L_{\alpha_i}$
  be the codimension $1$ face
  between $C_{i-1}$ and $C_i$ and
  $s_{\alpha_i}$ the
  corresponding reflection. Then
  $w = s_{\alpha_k} \circ \dots \circ s_{\alpha_2} \circ s_{\alpha_1}$ takes
  $C$ to $C'$.
\end{proof}

\begin{corollary}
  Let $\Pi, \Pi'$ be two families
  of simple roots. Then there
  exists $w \in W$ with
  $w(\Pi) = \Pi'$.
\end{corollary}

\begin{theorem}
  Let $R, R_+, \Pi = \{\alpha_1, \dots, \alpha_r\}$
  be fixed, and let
  $s_1, \dots, s_r$ be the reflections
  corresponding to $\alpha_1, \dots, \alpha_r$. Then
  \begin{enumerate}
    \item $s_1, \dots, s_r$ generates
      $W$.
    \item $W(\Pi) = R$.
  \end{enumerate}
\end{theorem}

\begin{proof}
  (2) Fix a set $\Pi$ of simple roots,
  and let $\alpha \in R$. Choose a
  Weyl chamber with a face $\subseteq L_\alpha$.
  Let $C_+$ be the Weyl chamber
  corresponding to $\Pi$, then
  there exists $w \in W$
  such that $w(C_+) = C_\alpha$,
  so $\alpha \in w(\Pi)$.

  (1) We claim that for any
  $s = s_\alpha \in W$,
  there exist $s_{i_1}, \dots, s_{i_\ell} \in \{s_1, \dots, s_r\}$
  such that
  $s_{i_1} \circ \dots \circ s_{i_\ell} = s$.
  Let $C = s(C_+)$, then we want
  \[
    (s_{i_1} \circ \dots \circ s_{i_\ell})(C_+) = C.
  \]
  We induct on the length of this chain.
  Suppose $L_\alpha = s_\beta L_\gamma$
  and
  assume that
  $s_\alpha, s_\beta$ are both
  products of simple reflections.
  Let $C_i$ be the chamber bounded
  by $L_\alpha$ and $L_\beta$, and
  let $C_{i + 1}$ be the chamber
  bounded by $L_\beta$ and $L_\gamma$.
  Then
  we can write
  \[
    s_\gamma
    = s_\beta \circ s_\alpha \circ s_\beta
  \]
  since $\alpha = s_\beta(\gamma)$.
  Thus $s_\gamma$ is also a product
  of simple reflections. So
  we have chains
  $C_{i + 1} \to C_+$,
  $L_\gamma \to L_i$, and
  $\gamma \to \alpha_i$, which
  proves the inductive step.
\end{proof}

\begin{corollary}
  $\langle s_1, \dots, s_r \rangle \Pi = R$.
\end{corollary}

\begin{definition}
  Let $w \in W$. Define the
  \emph{length}
  \begin{align*}
    \ell(w)
    &= \text{number of hyperplanes separating $C_+$ and $w(C_+)$} \\
    &= \text{number of roots $\alpha \in R_+$ with $w(\alpha) \in R_-$}.
  \end{align*}
\end{definition}

\begin{theorem}
  For $w \in W$, write
  $w = s_{i_1} \circ \dots \circ s_{i_\ell}$
  such that $\ell$ is minimal.
  Then $\ell = \ell(w)$.
\end{theorem}

\begin{proof}
  Let $C_0 = C_+$ and
  $C_k = (s_{i_1} \circ \dots \circ s_{i_k})(C_+)$.
  Let $\beta_k = (s_{i_1} \circ \dots \circ s_{i_{k - 1}})(\alpha_{i_k})$.
  Then
  \[
    C_k = s_{\alpha_{i_k}}(C_{k - 1})
  \]
  since
  $C_k = (s_{i_1} \circ \dots \circ s_{i_{k - 1}})(s_{i_k} C_+)$,
  $\beta_k = (s_{i_1} \circ \dots \circ s_{i_{k - 1}})(\alpha_{i_k})$,
  and $C_{k - 1} = (s_{i_1} \circ \dots \circ s_{i_{k - 1}})(C_+)$.
  This proves that
  $\ell(w) \le \ell$.
  On the other hand, we have
  $\beta_k \ne \pm \beta_j$, since
  otherwise for $j > k$,
  \[
    \alpha_{i_k}
    = \pm (s_{i_1} \circ \dots \circ s_{i_{j - 1}})(\alpha_{i_j}),
  \]
  so $s_{i_k} = (s_{i_1} \circ \dots \circ s_{i_{j - 1}}) \circ s_{i_j} \circ (s_{i_1} \circ \dots \circ s_{i_{j - 1}})^{-1} = s_{i_k} \circ \dots \circ s_{i_{j - 1}} \circ s_{i_j} \circ s_{i_{j - 1}} \circ \dots \circ s_{i_k}$, so
  \[
    w = s_{i_1} \circ \dots \circ s_{i_{k - 1}} \circ s_{i_{k}} \circ \dots \circ s_{i_{j - 1}} \circ (s_{i_j} \circ \dots \circ s_{i_{k + 1}}) \circ (s_{i_{k + 1}} \circ \dots \circ s_{i_j}) \circ \dots \circ s_{i_\ell}.
  \]
  The parts in parentheses can be
  canceled, contradicting the
  minimality of $\ell$.
  So $\ell(w) \ge \ell$.
\end{proof}

\begin{corollary}
  If $w(C_+) = C_+$, then $w = 1$.
\end{corollary}

\section{Root Lattices}
\begin{definition}
  Define the following lattices
  (note that these are in fact all
  lattices):
  \begin{enumerate}
    \item The \emph{root lattice}
      ($\subseteq E$)
      is the abelian group generated
      by the roots.
    \item The \emph{coroot lattice}
      ($\subseteq E^*$)
      is the abelian group generated
      by the coroots.
    \item The weight lattice is
      $\{\lambda \in E : (\lambda, \alpha) \in \Z\} \subseteq E$.
  \end{enumerate}
\end{definition}
