\chapter{Oct.~22 --- The Universal Enveloping Algebra}

\section{The Universal Enveloping Algebra, Continued}

\begin{example}
  Let $\mathfrak{sl}(2, \C)$ be generated
  by $\{e, f, h\}$, and
  \[
    C = \frac{1}{2} h^2 + ef + fe.
  \]
  This is known as the \emph{Casimir operator}.
  Note that $C$ commutes with
  $e, f, h$ and hence lies in the
  center of $\mathfrak{sl}(2, \C)$.
  We can explicitly check this for $e$
  as follows:
  \begin{align*}
    eC
    &= e^2 f + efe + \frac{1}{2} eh^2
    = e(fe + h) + (fe + h) e
    + \frac{1}{2} (he - 2e) h \\
    &= efe + fe^2 + \frac{1}{2} heh + eh + he - eh
    = efe + fe^2 + \frac{1}{2} h(he - 2e) + he \\
    &= efe + fe^2 + \frac{1}{2} h^2 e
    = \big(ef + fe + \frac{1}{2} h^2\big) e
    = Ce.
  \end{align*}
  Thus $\rho(C)$ for any representation
  $V$ of $\mathfrak{sl}(2, \C)$
  acts as a constant.
\end{example}

\begin{prop}
  We have the following:
  \begin{enumerate}
    \item The adjoint action of $\g$ on
      $\g$ can be extended to an action
      on $U(\g)$ by
      \[
        \ad_x(a b)
        = \ad_x(a) b + a \ad_x(b)
        \quad\text{and}\quad
        \ad_x a = xa - ax.
      \]
    \item Let $\mathcal{Z}(U(\g))$ be the
      center of $U(\g)$. Then
      $\mathcal{Z}(U(\g)) = (U(\g))^{\ad \g}$.
  \end{enumerate}
\end{prop}

\section{The Poincar\'e-Birkhoff-Witt Theorem}

\begin{remark}
  Let $\g$ be a finite-dimensional Lie
  algebra over a field $\K$, and let
  $U(\g)$ be its universal enveloping
  algebra. Note that we cannot put
  on a grading on $U(\g)$ since
  if we assign $\deg(x_i) = 1$ for
  $x_1, \dots, x_n$ a basis of $\g$,
  then $\deg(x_1 \dots x_\ell) = \ell$,
  but we have
  $[x_i, x_j] = \sum_k c_{ij}^k x_k$.

  Instead we can define a \emph{filtration}.
  Let $U_k(\g)$ be the subspace in $U(\g)$
  spanned by the products
  $x_{i_1} \cdots x_{i_p}$ for $p \le k$. Then
  \[
    \K = U_0(\g) \subseteq U_1(\g)
    \subseteq \cdots
    \subseteq U(\g) = \bigcup U_p(\g).
  \]
\end{remark}

\begin{prop}\label{prop:filtration}
  We have the following:
  \begin{enumerate}
    \item $U(\g)$ is a \emph{filtered algebra}, i.e.
      if $x \in U_p(\g)$, $y \in U_q(\g)$,
      then $xy \in U_{p + q}(\g)$.
    \item If $x \in U_p(\g)$, $y \in U_q(\g)$,
      then $xy - yx \in U_{p + q - 1}(\g)$.
    \item Let $x_1, \dots, x_n$ be an ordered basis
      in $\g$. Then the monomials
      $x_1^{k_1} \dots x_n^{k_n}$
      for $\sum k_i \le p$ span $U_p(\g)$.
  \end{enumerate}
\end{prop}

\begin{proof}
  $(1)$ This is obvious.

  $(2)$ We use induction on $p$.
  First let $p = 1$. Then we have
  \[
    x(y_1 \cdots y_q) - (y_1 \cdots y_q)x
    = \sum_i y_1 \cdots [x, y_i] \cdots y_q
  \]
  where $[x, y_i] = xy_i - y_i x$.
  So the above expression lies in
  $U_q(\g)$. Now assume it is true for $p$.
  Then
  \[
    x_1 \cdots x_{p + 1} y
    \equiv x_1 \cdots x_p y x_{p + 1}
    \equiv y x_1 \cdots x_p x_{p + 1}
    \pmod{U_{p + q}(\g)}.
  \]
  So the result is true for $p + 1$.
  
  $(3)$ We again induct on $p$. The case
  $p = 1$ is obvious. Note that
  $U_{p + 1}(\g)$ is generated by
  $xy$ for $x \in \g$ and $y \in U_p(\g)$.
  By the induction hypothesis, $y$ can
  be represented as such a sum of monomials
  \[
    x_i (x_1^{k_1} \cdots x_i^{k_i} \cdots x_n^{k_n})
    \equiv x_1^{k_1} \cdots x_i^{k_i + 1} \cdots x_n^{k_n}
    \pmod{U_p(\g)}.
  \]
  This proves the claim.
\end{proof}

\begin{corollary}
  Each of the $U_p(\g)$ are finite-dimensional.
\end{corollary}

\begin{corollary}
  Define $\Gr U(\g) = \bigoplus_p U_p(\g) / U_{p - 1}(\g)$.
  This is a graded commutative algebra.
\end{corollary}

\begin{theorem}[Poincar\'e-Birkhoff-Witt]
  Let $x_1, \dots, x_n$ be an ordered basis
  in $\g$. Then the monomials
  as in Proposition \ref{prop:filtration}(3)
  form a basis in $U(\g)$.
\end{theorem}

\begin{proof}
  We need to show linear independence.
  The idea is to construct a representation $V$
  so that the operators corresponding to these
  monomials are linearly independent.
  We can take
  $V = U(\g) \cdot 1$ with
  \[
    \rho(x_i) x_{j_1} \cdots x_{j_n}
    = x_i x_{j_1} \cdots x_{j_n}
  \]
  for $i \le j_1, \dots, j_n$.
  For $i > j_1$, note that we have
  \[
    \rho(x_2) x_1 = \rho(x_2)(\rho(x_1)) \cdot 1
    = \rho(x_1) \rho(x_2) \cdot 1
    + \rho([x_2, x_1]) \cdot 1
    = x_1 x_2 + \sum a_i x_i.
  \]
  We can define the general case in a
  similar way.
\end{proof}

\begin{theorem}[Poicar\'e-Birkhoff-Witt, alternative]
  $\Gr U(\g)$ is isomorphic to the symmetric
  algebra $S(\g)$. There is a well-defined
  map
  \begin{align*}
    S^p(\g) &\longrightarrow \Gr^p(\g) \\
    a_1 \cdots a_p
    &\longmapsto
    a_1 \cdots a_p \Mod{U_{p - 1}(\g)}.
  \end{align*}
  with well-defined inverse given by
  \begin{align*}
    \Gr^p(\g) &\longrightarrow S^p(\g) \\
    a_1 \dots a_p
    &\longmapsto
    a_1 \cdots a_p \\
    a_1 \dots a_e
    &\longmapsto
    0, \quad e < p.
  \end{align*}
\end{theorem}

\begin{corollary}
  The map $\g \to U(\g)$ is injective.
\end{corollary}

\begin{corollary}
  Let $\g_1, \g_2 \subseteq \g$ be subalgebras.
  Then $\g = \g_1 \otimes \g_2$ as
  vector space, and
  \[
    U(\g) \otimes U(\g_1) \longrightarrow U(\g_2)
  \]
  is an isomorphism of vector spaces.
\end{corollary}

\begin{corollary}
  $U(\g)$ has no zero divisors.
\end{corollary}

\begin{theorem}
  The map $S(\g) \to U(\g)$ which sends
  \[
    \sym(x_1 \cdots x_p)
    = \frac{1}{p} \sum_{s \in S_p}
    x_{s(1)} \cdots x_{s(p)}
  \]
  is an isomorphism of $\g$-modules
  (with respect to the adjoint action).
\end{theorem}

\section{Ideals and Subalgebras}
\begin{remark}
  Recall that $\mathfrak{h} \subseteq \g$
  is an ideal if $[x, y] \in \mathfrak{h}$
  for every $x \in \g$, $y \in \mathfrak{h}$.
\end{remark}

\begin{lemma}
  If $f : \g_1 \to \g_2$ is a morphism,
  then $\ker f$ is an ideal of $\g_1$,
  $\im f$ is a subalgebra of $\g_2$,
  and $\g_1 / {\ker f} \cong \im f$.
\end{lemma}

\begin{lemma}
  Let $I_1, I_2$ be ideals in $\g$. Define
  \begin{enumerate}
    \item $I_1 + I_2 = \{x_1 + x_2 :
      x_1 \in I_1, x_2 \in I_2\}$.
    \item $[I_1, I_2]$ is the subspace
      spanned by $[x, y]$ for $x \in I_1$,
      $y \in I_2$.
      \item $I_1 \cap I_2$.
  \end{enumerate}
  Then (1)-(3) are all idals in $\g$.
\end{lemma}

\begin{definition}
  The \emph{commutant} of a Lie algebra
  $\g$ is
  the ideal $[\g, \g]$.
\end{definition}

\begin{lemma}
  $\g / [\g, \g]$ is an abelian Lie algebra.
  Moreover, $[\g, \g]$ is the smallest
  ideal with the property that $\g / I$
  is an abelian Lie algebra.
\end{lemma}

\begin{example}
  $[\gl(n, \K), \gl(n, \K)] = [\mathfrak{sl}(n, \K), \mathfrak{sl}(n, \K)] = \mathfrak{sl}(n, \K)$.
  To see this, let $z = [x, y]$, so
  we have $\tr z = 0$. Note that
  $E_{i, i} - E_{j, j} = [E_{i, j}, E_{j, i}]$,
  where $E_{i, j}$ contains a single
  $1$ in the $(i, j)$ position and $0$
  otherwise, so
  $2E_{i, j} = [E_{i, i} - E_{j, j}, E_{i, j}]$.
\end{example}

\section{Solvable and Nilpotent Lie Algebras}
\begin{definition}
  Let $\g$ be a Lie algebra. The
  \emph{derived series} $D^i \g$ of
  $\g$ is defined by
  \[
    D^0 \g = \g
    \quad \text{and} \quad D^{i + 1} \g = [D^i \g, D^i \g].
  \]
\end{definition}

\begin{remark}
  Each $D^i \g$ is an ideal in $\g$
  and $D^i \g / D^{i + 1} \g$ is abelian.
\end{remark}

\begin{prop}
  The following conditions are equivalent:
  \begin{enumerate}
    \item $D^n \g = 0$ for large enough $n$.
    \item There exists a sequence of subalgebras
      $\g = \mathfrak{a}^0 \supseteq \mathfrak{a}^1 \supseteq \cdots
      \supseteq \mathfrak{a}^n = \{0\}$
      such that $\mathfrak{a}^{i + 1}$ is an ideal
      in $\mathfrak{a}^i$ and
      the quotient $\mathfrak{a}^i / \mathfrak{a}^{i + 1}$
      is abelian.
    \item For large enough $n$, all commutators
      of the form
      \[
        [\dots, [[x_1, x_2], [x_3, x_4]], \dots,]
      \]
      ($2^n$ terms) is zero.
  \end{enumerate}
\end{prop}

\begin{proof}
  $(1 \Leftrightarrow 2)$ This is clear.

  $(1 \Rightarrow 2)$ Take
  $\mathfrak{a}^i = D^i \g$.

  $(2 \Rightarrow 1)$ If
  $\mathfrak{a}^i$ satisfies (2), then
  $\mathfrak{a}^{i + 1} \supseteq [\mathfrak{a}^i, \mathfrak{a}^i]$, so
  by induction we have (1).
\end{proof}

\begin{definition}
  A Lie algebra $\g$ is called
  \emph{solvable} if it satisfies one of
  the above conditions.
\end{definition}

\begin{definition}
  Define the \emph{lower central series} $D_i \g \subseteq \g$
  by
  \[
    D_0\g = \g
    \quad \text{and} \quad
    D_{i + 1} \g = [\g, D_i \g].
  \]
\end{definition}

\begin{prop}
  The following conditions are equivalent:
  \begin{enumerate}
    \item $D_n \g = 0$ for large enough $n$.
    \item There exists a sequence of
      ideals such that
      $\g = \mathfrak{a}_0 \supseteq \mathfrak{a}_1 \supseteq \cdots \supseteq \mathfrak{a}_n = \{0\}$
      such that $[y, \mathfrak{a}_i] \subseteq \mathfrak{a}_{i + 1}$.
    \item For large enough $n$, all
      commutators of the form
      \[
        [[\dots, [[x_1, x_2], x_3], x_4], \dots x_n]
      \]
      ($n$ terms) are zero.
  \end{enumerate}
\end{prop}

\begin{definition}
  A Lie algebra $\g$ is called
  \emph{nilpotent} if it satisfies one of
  the above conditions.
\end{definition}

\begin{example}
  The Lie algebra of strictly
  lower triangular matrices is nilpotent.
  The non-strictly lower triangular
  matrices form a solvable Lie algebra.
\end{example}
