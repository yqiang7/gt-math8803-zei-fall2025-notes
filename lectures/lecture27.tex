\chapter{Nov.~24 --- Representations of Semisimple Lie Algebras}

\section{Highest Weight Representations}

\begin{definition}
  A representation $V$ of $\g$
  is called a
  \emph{highest weight representation}
  with highest weight
  $\lambda$ if it is generated by
  a given highest weight vector of
  weight $\lambda$.
\end{definition}

\begin{prop}
  Any finite-dimensional
  representation $V$ contains a nonzero
  highest weight vector with some
  weight $\lambda$.
\end{prop}

\begin{corollary}
  Any irreducible finite-dimensional
  representation of $\g$ is a
  highest weight representation.
\end{corollary}

\begin{proof}
  Let $P(V)$ be the set of weights,
  which is finite. Define
  \[
    \rho^\vee
    = \sum_{i = 1}^r \omega_i^\vee,
  \]
  where $\omega_i^\vee$ are the
  coroots defined by
  $\langle \omega_i^\vee, \alpha_j \rangle = \delta_{i, j}$.
  Now pick $\lambda$ such that
  $(\lambda, \rho^\vee)$ is maximal.
  Then
  \[
    (\lambda + \alpha_i, \rho^\vee)
    = (\lambda, \rho^\vee) + 1,
  \]
  hence $\lambda + \alpha_i \notin P(V)$
  by the maximality of
  $(\lambda, \rho^\vee)$.
  Thus for every $v \in V[\lambda]$,
  we have $e_i v = 0$.
\end{proof}

\begin{definition}
  Consider the ideal
  $I_\lambda \subseteq U(\g)$ (the
  universal enveloping algebra)
  generated by $h - \lambda(h) 1$
  for some $h \in \mathfrak{h}$
  (the Cartan subalgebra), and
  $e_i$ for $i = 1, \dots, r$.
  Then the \emph{Verma module} is
  \[
    M_\lambda = U(\g) / I_\lambda.
  \]
\end{definition}

\begin{remark}
  Recall that we had a
  decomposition
  $\g = \mathfrak{n}_- \oplus \mathfrak{h} \oplus \mathfrak{n}_+$,
  where $\mathfrak{h}$ is
  the Cartan subalgebra,
  $\mathfrak{n}_+$ contains the
  positive root spaces generated by
  $\{e_i\}$, and $\mathfrak{n}_-$
  contains the negative root spaces
  generated by $\{f_i\}$.
\end{remark}

\begin{prop}
  Define a map
  $\phi : U(\mathfrak{n}_-) \to M_\lambda$
  given by $\phi(x) = x v_\lambda$,
  where $v_\lambda$ is the class
  of $1$ in $U(\g) / I_\lambda$.
  Then $\phi$ is an isomorphism of
  left $U(\mathfrak{n}_-)$-modules.
\end{prop}

\begin{proof}
  By the Poincar\'e-Birkhoff-Witt
  theorem, we have an
  isomorphism
  \[
    \xi : U(\mathfrak{n}_-) \otimes U(\mathfrak{h} \oplus \mathfrak{n}_+)
    \longrightarrow U(\g)
  \]
  where $\xi^{-1}(I_\lambda) = U(\mathfrak{n}_-) \otimes K_\lambda$
  for
  $K_\lambda = \sum_i U(\mathfrak{h} \oplus \mathfrak{n}_+)(h_i - \lambda(h_i) 1)
  + \sum_i U(\mathfrak{h} \oplus \mathfrak{n}_+) e_i$.
  Thus
  \pagebreak
  \[
    K_\lambda
    = \ker(\lambda_+ : U(\mathfrak{h} \oplus \mathfrak{n}_+) \to \C),
  \]
  where $\lambda_+(h) = \lambda(h)$
  for $h \in \mathfrak{h}$
  and $\lambda_+(e_i) = 0$.
  Thus $U(\mathfrak{n}_-) = (U(\mathfrak{n}_-) \otimes U(\mathfrak{h} \oplus \mathfrak{n}_+)) / K_\lambda \to U(\g)$.
\end{proof}

\begin{remark}
  Note that
  $M_\lambda$ is
  an induced module
  $U(\g) \otimes_{U(\mathfrak{h} \oplus \mathfrak{n}_+)} \C_\lambda$,
  where $\C_\lambda$ is the
  $1$-dimensional representation of
  $\mathfrak{h} \oplus \mathfrak{n}_+$
  acting using $\lambda_+$.
\end{remark}

\begin{corollary}
  $M_\lambda$ has weight
  decomposition
  $P(M_\lambda) = \lambda - R_+$,
  where
  \[
    R_+
    = \left\{\sum_{i = 1}^r k_i \alpha_i : k_i \in \N\right\}.
  \]
  Moreover, $\dim M_\lambda[\lambda] = 1$, and
  each weight subspace is
  $1$-dimensional.
\end{corollary}

\begin{prop}[Universal property of $M_\lambda$]
  If $V$ is a representation of
  $\g$ and $v \in V$ is a vector with
  \begin{align*}
    hv &= \lambda(h) v, \quad h \in \mathfrak{h}, \\
    e_i v &= 0, \quad \quad 1 \le i \le r,
  \end{align*}
  then there exists a unique
  homomorphism $\eta : M_\lambda \to V$
  such that $\eta(v_\lambda) = v$.
  If $V$ is generated by $v$, then
  $V$ is a quotient of $M_\lambda$.
\end{prop}

\begin{proof}
  Left as an exercise, but consider
  $\widetilde{\eta} : U(\g) \to V$
  given by $\widetilde{\eta}(x) = x v$.
\end{proof}

\begin{remark}
  Every highest weight representation
  has a unique highest weight
  up to scaling.
\end{remark}

\begin{prop}
  For every $\lambda \in \mathfrak{h}^*$,
  $M_\lambda$ has a unique
  irreducible quotient $L_\lambda$.
  Moreover, $L_\lambda$ is a quotient
  of every highest weight module $V$ 
  with highest weight $\lambda$.
\end{prop}

\begin{proof}
  Consider a proper submodule
  $Y \subseteq M_\lambda$, which
  cannot contain $v_\lambda$. Then
  \[
    P(Y)
    \subseteq (\lambda - R_+) \setminus \{\lambda\}.
  \]
  Let $J_\lambda$ be the union of
  such submodules $Y$, and define
  $L_\lambda = M_\lambda / J_\lambda$.
  If $V$ is any other quotient, then
  we have a natural map
  $M_\lambda \to V$. Then
  $K = \ker(M_\lambda \to V) \subseteq J_\lambda$,
  so the statement follows.
\end{proof}

\begin{corollary}
  The irreducible highest weight
  modules are classified by the
  highest weight $\lambda \in \mathfrak{h}^*$, i.e.
  there is a bijection $\lambda \mapsto L_\lambda$.
\end{corollary}

\section{Finite-Dimensional Modules}

\begin{definition}
  Define the set $P_+ \subseteq P$
  of \emph{dominant integral weights}
  by
  \[
    P_+ =
    \left\{
      \sum m_i \omega_i : m_i \ge 0
    \right\}
    = \left\{
      \lambda \in P :
      \langle \lambda, \alpha_i^\vee \rangle \in \Z_+
    \right\}.
  \]
  Also define $P_F$ to be the
  set of weights for which
  $L_\lambda$ is finite-dimensional.
\end{definition}

\begin{prop}
  $P_F \subseteq P_+$.
\end{prop}

\begin{proof}
  Consider $(\mathfrak{sl}_2)_i$,
  then the highest weight $\lambda \in V$
  must have
  $\langle \lambda, \alpha_i^\vee \rangle$
  non-negative and integer.
\end{proof}

\begin{lemma}
  If $\lambda \in P_+$, then
  in $L_\lambda$, we have
  $f_i^{\lambda(h_i) + 1} v_\lambda = 0$.
\end{lemma}

\begin{proof}
  Consider $(\mathfrak{sl}_2)_i$.
  Then we have
  \begin{align*}
    e_i f_i^{\lambda(h_i) + 1} v_\lambda &= 0, \\
    e_j f_i^{\lambda(h_i) + 1} v_\lambda &= 0, \quad j \ne i.
  \end{align*}
  Then $f_i^{\lambda(h_i) + 1} v_\lambda$
  is a highest weight vector
  generating a proper submodule, so
  it must be $0$ in $L_\lambda$.
\end{proof}

\begin{lemma}\label{lem:weyl-symmetry}
  Let $V$ be a $\g$-module with
  weight decomposition into
  finite-dimensional subspaces.
  If $V$ is a sum of
  $(\mathfrak{sl}_2)_i$-modules
  for any $i = 1, \dots, r$, then
  for every $\lambda \in P$ and
  $w \in W$ (the Weyl group),
  \[
    \dim V[\lambda]
    = \dim V[w(\lambda)].
  \]
\end{lemma}

\begin{proof}
  It is enough to show this for
  $s_i \in W$, since the simple
  reflections $s_i$ generate $W$.
  Moreover, it is enough to prove
  $\dim V[\lambda] \le \dim V[s_i(\lambda)]$
  since $s_i^2 = 1$.
  Consider $(\mathfrak{sl}_2)_i$.
  If $(\lambda, \alpha_i^\vee) = m \ge 0$,
  then
  \[
    f_i^m : V[\lambda] \to V[s_i(\lambda)],
  \]
  which is always injective.
  For $m < 0$, repeat the same
  argument using $e_i^m$.
\end{proof}

\begin{theorem}
  For any $\lambda \in P_+$,
  $L_\lambda$ is finite-dimensional.
  Therefore, $P_F = P_+$, so the
  irreducible finite-dimensional
  representations of $\g$ are
  classified by their highest
  weight $\lambda \in P_+$.
  Moreover, for any $\mu \in P$ and
  $w \in W$, we have
  $\dim L_\lambda[\mu] = \dim L_\lambda[w(\mu)]$.
\end{theorem}

\begin{proof}
  Since $f_i^{\lambda(h_i) + 1} v_\lambda = 0$,
  we know
  $v_\lambda$ generates an
  irreducible $(\mathfrak{sl}_2)_i$-module
  with highest weight $\lambda(h_i)$,
  and every non-zero vector
  in $V$ generates an
  $(\mathfrak{sl}_2)_i$-module. Then
  we can write 
  $L_\lambda$ as a linear combination
  of $a_1 \dots a_N V_\lambda$
  where $a_i \in \g$. By Lemma
  \ref{lem:weyl-symmetry},
  $P(W_\lambda)$ is  is
  $W$-invariant. Thus for
  any $\mu \in P(L_\lambda) \cap P_+$,
  we can write $\mu = \lambda - \beta$
  for some $\beta \in R_+$. Then we have
  \[
    \langle \mu, \rho^\vee \rangle
    = \langle \lambda, \rho^\vee \rangle
    - \langle \beta, \rho^\vee \rangle
    \le \langle \lambda, \rho^\vee \rangle.
  \]
  Write $\mu = \sum_{i = 1}^r m_i \omega_i$ for
  $m_i \in \Z_+$. Then we have
  \[
    \sum_i m_i (\omega_i, \rho^\vee)
    \le (\lambda, \rho^\vee).
  \]
  Since $\langle \omega_i, \rho^\vee \rangle \ge 1 / 2$
  (as $\rho^\vee = \frac{1}{2} \sum_{\alpha \in R_+} \alpha^\vee$),
  we know $P(L_\lambda) \cap P_+$ is
  finite.
  Since $WP_+ = P$, we have
  \[
    W(P(L_\lambda) \cap P_+)
    = P(L_\lambda)
  \]
  since $P(L_\lambda)$ is $W$-invariant.
  Thus it is finite-dimensional.
  The last statement is by
  $W$-invariance.
\end{proof}

\section{Weyl Character Formula}

\begin{remark}
  Recall that if $V$ is a
  $\g$-module, then we can consider
  the character
  $\chi_V(g) = {\tr}|_V(g)$
  for $g \in G$ in the
  corresponding Lie group.
  Let $\mathfrak{h} \subseteq \g$
  be the Cartan subalgebra, and
  consider $\chi_V(e^h)$
  for $h \in \mathfrak{h}$.

  If $V = \bigoplus_{\mu \in P} V[\mu]$,
  then the character is given by
  \[
    \chi_V(e^h)
    = \sum_{\mu \in P} (\dim V[\mu]) e^{\mu(h)}.
  \]
  \pagebreak
  Consider the polynomial algebra
  $\Z[P]$, which sits inside
  the algebra of analytic functions
  on $\mathfrak{h}$. Consider
  $\lambda \mapsto e^\lambda$ and
  define $e^\lambda(h) := e^{\lambda(h)}$.
  Then for $\chi_V \in \Z[P]$, we have
  \[
    \chi_V = \sum_{\mu \in P}
    (\dim V[\mu]) e^\mu.
  \]
  where the above is the generating
  function for the dimensions of
  the weight subspaces.
\end{remark}

\begin{definition}
  The \emph{category $\mathcal{O}$}
  is the category of representations
  $V$ of $\g$ with weight
  decomposition into finite-dimensional
  weight spaces $V = \bigoplus_{\mu \in P} V[\mu]$
  such that $P(V)$ is contained
  in a union of sets $\lambda^i - R_+$
  for a finite collection of
  weights $\lambda^i \in P$.
\end{definition}

\begin{example}
  Let $V = M_\lambda = U(\mathfrak{n}_-)v_\lambda$ be the Verma
  module. We can write
  \[
    U(\mathfrak{n}_-)
    = \bigotimes_{\alpha \in R_+} \C[e_{-\alpha}]
  \]
  by the Poincar\'e-Birkhoff-Witt
  theorem. Then we have
  \[
    \sum_{\mu} U(\mathfrak{n}_-)[\mu]
    e^\mu
    = \frac{1}{\prod_{\alpha \in R_+}(1 - e^\alpha)}.
  \]
  Thus the character of $M_\lambda$ is
  \[
    \chi_{M_\lambda}
    = \frac{e^{\lambda + \rho}}
    {\prod_{\alpha \in R_+}(e^{\alpha / 2} - e^{-\alpha / 2})},
  \]
  where $\Delta = \prod_{\alpha \in R_+} (e^{\alpha / 2} - e^{-\alpha / 2})$
  is known as the \emph{Weyl denominator}.
\end{example}

\begin{remark}
  Now consider
  \begin{align*}
    \epsilon : W
    &\longmapsto \Z / 2\Z \\
    \omega
    &\longmapsto
    \det(\omega|_\mathfrak{h}) =
    (-1)^{\ell(\omega)},
  \end{align*}
  where $\ell(\omega)$ is the
  length of $\omega$ (note that
  $\det(s_i) = -1$). For
  $A_{n - 1}$, this is the sign
  of the permutation.
\end{remark}

\begin{definition}
  We say $f \in \C[P]$ is
  \emph{anti-invariant} if
  $\omega(f) = (-1)^{\ell(\omega)} f$.
\end{definition}

\begin{theorem}[Weyl character formula]
  For every $\lambda \in P_+$, the
  character $\chi_\lambda = \chi_{L_\lambda}$
  is given by
  \[
    \chi_V
    = \frac{\sum_{w \in W} (-1)^{\ell(w)} e^{\lambda + \rho}}{\Delta}.
  \]
\end{theorem}

\begin{corollary}[Weyl denominator formula]
  For $\lambda = 0$, we have
  $\Delta = \sum_{w \in W} (-1)^{\ell(w)} e^{w(\rho)}$.
\end{corollary}

\begin{remark}[Dimension formula]
  The Weyl character formula says that
  \[
    \chi_\lambda(e^h)
    = \frac{\sum_{w \in W} (-1)^{\ell(w)} e^{\langle w(\lambda + \rho), h \rangle}}
    {\prod_{\alpha \in R_+} (e^{\langle \alpha, h \rangle / 2} - e^{-\langle \alpha, h \rangle / 2})}.
  \]
  For $h = 2th_\rho$ where
  $h_\rho \leftrightarrow \rho \in \mathfrak{h}^*$, the above becomes
  \[
    \chi_\lambda(e^{2 t h_\rho})
    = \frac{\sum_{w \in W} (-1)^{\ell(w)} e^{2 t \langle w(\lambda + \rho), \rho \rangle}}
    {\prod_{\alpha \in R_+} (e^{t \langle \alpha, \rho \rangle} - e^{-t \langle \alpha, \rho \rangle})}
    = \frac{\prod_{\alpha \in R_+} (e^{t\langle \alpha, \lambda + \rho \rangle} - e^{-t\langle \alpha, \lambda + \rho \rangle})}
    {\prod_{\alpha \in R_+} (e^{t \langle \alpha, \rho \rangle} - e^{-t \langle \alpha, \rho \rangle})},
  \]
  where the last equality is
  by the Weyl denominator formula.
  Taking $t \to 0$, we get the
  \emph{dimension formula}
  \[
    \dim L_\lambda
    = \frac{\prod_{\alpha \in R_+} \langle \alpha, \lambda + \rho \rangle}
    {\prod_{\alpha \in R_+} \langle \alpha, \rho \rangle}.
  \]
\end{remark}
