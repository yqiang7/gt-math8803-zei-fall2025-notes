\chapter{Oct.~20 --- Representations of \texorpdfstring{$\mathfrak{sl}(2, \C)$}{sl(2, C)}}

\section{Irreducible Representations of \texorpdfstring{$\mathfrak{sl}(2, \C)$}{sl(2, C)}}
\begin{remark}
  Consider $\mathfrak{sl}(2, \C), \SL(2, \C), \mathfrak{su}(2), \SU(2)$.
  Recall from last time that the categories
  of complex representations
  of these Lie groups and Lie algebras
  are equivalent and completely reducible.

  We will now study (finite-dimensional)
  irreducible representations of
  $\mathfrak{sl}(2, \C)$. Recall that
  $\mathfrak{sl}(2, \C)$ has a basis
  given by
  \[
    h = \begin{pmatrix}
      1 & 0 \\
      0 & -1
    \end{pmatrix}, \quad
    e = \begin{pmatrix}
      0 & 1 \\
      0 & 0
    \end{pmatrix}, \quad
    f = \begin{pmatrix}
      0 & 0 \\
      1 & 0
    \end{pmatrix},
  \]
  with relations
  $[e, f] = h$, $[h, e] = 2e$, and
  $[h, f] = -2f$.
\end{remark}

\begin{definition}
  Let $V$ be a representation of
  $\mathfrak{sl}(2, \C)$. A vector
  $v \in V$ is called a
  \emph{vector of weight $\lambda$}
  (for $\lambda \in \C$) if it is an
  eigenvector for $h$ with eigenvalue
  $\lambda$, i.e. $hv = \lambda v$.
  Denote by $V[\lambda]$ the subspace of
  vectors with eigenvalue $\lambda$.
\end{definition}

\begin{lemma}\label{lem:weight-shift}
  $e V[\lambda] \subseteq V[\lambda + 2]$
  and $f V[\lambda] \subseteq V[\lambda - 2]$.
\end{lemma}

\begin{proof}
  Let $v \in V[\lambda]$. Then
  $hv = \lambda$, so we have
  \[
    hev = ehv + [h, e] v
    = ehv + 2e v
    (\lambda + 2) ev,
  \]
  so $ev \in V[\lambda + 2]$. This
  proves the first statement, and
  one can prove the second similarly.
\end{proof}

\begin{theorem}
  Every finite-dimensional representation
  $V$ of $\mathfrak{sl}(2, \C)$
  can be written in the form
  \[
    V = \bigoplus_\lambda V[\lambda].
  \]
  Such a decomposition is called the
  \emph{weight decomposition} of $V$.
\end{theorem}

\begin{proof}
  Since every representation for
  $\mathfrak{sl}(2, \C)$ is
  completely reducible, it suffices to
  prove the statement for irreps.
  So let $V$ be irreducible.
  Let $V' = \sum_\lambda V[\lambda] = \bigoplus_\lambda V[\lambda]$.
  By Lemma \ref{lem:weight-shift}, $V'$ is
  stable under the
  actions of $e, f$, so it is a subrepresentation
  of $V$. Since $V$ is irreducible,
  $V' = V$.
\end{proof}

\begin{remark}
  Let $\lambda$ be a weight of $V$
  (i.e. $V[\lambda] \ne 0$) which is
  maximal in the following sense:
  \[
    \re \lambda
    \ge \re \lambda' \quad
    \text{for all weights } \lambda' \text{ of $V$}.
  \]
  Then the corresponding eigenvector
  $v \in V[\lambda]$ is called the
  \emph{highest weight vector}, and it
  always exists for finite-dimensional
  $V$.
\end{remark}

\begin{lemma}\label{lem:formulas-highest-weight}
  Assume $v \in V[\lambda]$ is the
  highest weight vector. Then
  \begin{enumerate}
    \item $ev = 0$.
    \item $v^k := f^k v / k!$
      (for $k \ge 0$) satisfies the following
      formulas:
      \[
        hv^k = (\lambda - 2k)v^k, \quad
        ev^k = (\lambda - k + 1) v^{k - 1}, \quad
        f v^k = (k + 1) v^{k + 1}.
      \]
  \end{enumerate}
\end{lemma}

\begin{proof}
  We prove the second formula in (2), the
  rest is left as an exercise. We induct
  on $k$. For $k = 1$,
  \[
    ev^1 = ef v = [e, f] v + fe v
    = hv = \lambda v.
  \]
  Now assume it is true for $k$, and
  we prove the formula for $k + 1$. We
  have
  \begin{align*}
    e v^{k + 1}
    = \frac{1}{k + 1} ef v^k
    &= \frac{1}{k + 1}(hv^k + fe v^k)
    = \frac{1}{k + 1}((\lambda - 2k)v^k + (\lambda - k + 1) f v^{k - 1}) \\
    &= \frac{1}{k + 1}(\lambda - 2k + (\lambda - k + 1) k)v^k
    = (\lambda - k) v^k,
  \end{align*}
  which is the corresponding formula for
  $k + 1$.
\end{proof}

\begin{lemma}
  Let $\lambda \in \C$. Define
  $M_\lambda$ (the \emph{Verma module})
  to be the infinite-dimensional vector space
  with basis $v^0, v^1, v^2, \dots$. Then
  \begin{enumerate}
    \item The formulas from
      Lemma \ref{lem:formulas-highest-weight}(2)
      and $ev^0 = 0$ define the structure of
      an infinite-dimensional representation
      on $M_\lambda$.
    \item If $V$ is a finite-dimensional
      irreducible module of $\mathfrak{sl}(2, \C)$
      which contains a nonzero highest
      weight vector of highest weight
      $\lambda$, then $V = M_\lambda / W$,
      where $W$ is some subrepresentation.
  \end{enumerate}
\end{lemma}

\begin{proof}
  This follows from Lemma
  \ref{lem:formulas-highest-weight}.
\end{proof}

\begin{theorem}
  We have the following:
  \begin{enumerate}
    \item For any $n \in \Z_{\ge 0}$, let
      $V_n$ be the finite-dimensional
      vector space with basis
      $v^0, v^1, \dots, v^n$. Define an
      action of $\mathfrak{sl}(2, \C)$ on
      $V$ by the formulas from
      Lemma \ref{lem:formulas-highest-weight}(2),
      with $f v^n = 0$ and $ev^0 = 0$.
      Then $V_n$ is an irreducible
      representation.
    \item For $n \ne m$, the representations
      $V_n, V_m$ are non-isomorphic.
    \item Every finite-dimensional irrep
      of $\mathfrak{sl}(2, \C)$ is
      isomorphic to one of the
      $V_n$.
  \end{enumerate}
\end{theorem}

\begin{proof}
  (1) Start with the Verma module $M_\lambda$
  for $\lambda = n$. Consider
  the subspace
  \[
    M' = \Span\{v^{n + 1}, v^{n + 2}, \dots\}.
  \]
  Note that $ev^{n + 1} = (n + 1 - (n + 1))v^n = 0$.
  Then $M_n / M'$ is a finite-dimensional
  representation with basis
  $v^0, v^1, \dots, v^n$, so it is
  isomorphic to $V_n$. Now we show that
  it is irreducible: If there is a
  subrepresentation $V' = \Span\{v_i\}$
  where $\{v_i\} \subseteq \{v^0, \dots, v^n\}$,
  then $V' = V_n$ since otherwise
  $V'$ is not invariant under the action of
  $e, f, h$. Thus
  $V_n$ is irreducible.

  (2) This follows since $V_n$ and
  $V_m$ have different dimensions.
\end{proof}

\begin{remark}
  The actions of $f$ and $e$ have the
  following actions on the $V_n$ ($f$ on
  top, $e$ on bottom):
  \begin{center}
  \begin{tikzcd}
    V_n \ar[r, "1", bend right=30, swap] & V_{n - 1} \ar[r, "2", bend right=30, swap] \ar[l, "n", bend right=30, swap] & V_{n - 2} \ar[r, "3", bend right=30, swap] \ar[l, "n - 1", bend right=30, swap] & \cdots \ar[r, "n - 1", bend right=30, swap] \ar[l, "n - 2", bend right=30, swap] & V_1 \ar[r, "n", bend right=30, swap] \ar[l, "2", bend right=30, swap] & V_0 \ar[l, "1", bend right=30, swap]
  \end{tikzcd}
  \end{center}
  Note that the eigenvalues of $h$
  are $n, n - 2, \dots, -n$.

  For $\SU(2)$, we have
  $J_z = \frac{1}{2} h$,
  with eigenvalues
  $\frac{n}{2}, \dots, -\frac{n}{2}$.
  This is called the \emph{spin}
  in quantum mechanics.
  We have seen that any representation
  of $\mathfrak{su}(2)$ lifts to
  a representation of $\SU(2)$.
  A representation of $\mathfrak{so}(3, \R)$
  will lift to a representation of
  $\SO(3, \R)$ if and only if
  the spin is an integer.
\end{remark}

\begin{theorem}
  Let $V$ be any finite-dimensional
  complex representation of
  $\mathfrak{sl}(2, \C)$. Then
  \begin{enumerate}
    \item $V$ admits a weight
      decomposition with respect to
      integer weights:
      $V = \bigoplus_{n \in \Z} V[n]$.
    \item $\dim V[n] = \dim V[-n]$, and
      $e^n : V[n] \to V[-n]$,
      $f^n : V[-n] \to V[n]$
      are isomorphisms.
  \end{enumerate}
\end{theorem}

\section{The Universal Enveloping Algebra}

\begin{definition}
  Let $\g$ be a Lie algebra over some field
  $\K$. The
  \emph{universal enveloping algebra}
  $U(\g)$ of $\g$ over $\K$ is the
  associative algebra with unit
  over $\K$ with generators
  $i(x)$ for $x \in \g$,
  subject to the relations
  \begin{enumerate}
    \item $i(x) + i(y) = i(x + y)$,
    \item $i(cx) = ci(x)$ for $c \in \K$,
    \item $i(x) i(y) - i(y) i(x) = i([x, y])$.
  \end{enumerate}
\end{definition}

\begin{remark}
  We will see that the map
  $i : \g \to U(\g)$ is injective, so
  we will just write $x$ instead of $i(x)$.
\end{remark}

\begin{remark}
  Without the condition (3), this is just
  the tensor algebra
  $T \g = \bigoplus_{n \ge 0} g^{\otimes n}$.
  So we can say
  \[
    U(\g) = T \g / \{xy - yx - [x, y]\}.
  \]
\end{remark}

\begin{example}
  Let $\g$ be a commutative Lie algebra.
  Then $U(\g)$ is generated by $x \in \g$
  subject to conditions
  $\{xy = yx\}$. Thus $U(\g) = S(\g)$, the
  symmetric algebra on $\g$. One can view $S(\g)$ as polynomials on the dual space $\g^*$. From this perspective, $S(\g) = \K[x_1, \dots, x_n]$, where $x_1, \dots, x_n$ form a basis of $\g$.
  In particular, this shows that
  $U(\g)$ is infinite-dimensional.
\end{example}

\begin{example}
  Recall that $\mathfrak{sl}(2, \C)$
  is the associative algebra generated
  by $1, e, f, h$ subject to
  \[
    \begin{cases}
      ef - fe = h, \\
      he - eh = 2e, \\
      hf - fh = -2f.
    \end{cases}
  \]
  Note that the product $h^m e^k f^\ell$
  is \emph{not} a matrix multiplication (we have
  $e^2 = f^2 = 0$ as matrices).
\end{example}

\begin{theorem}
  Let $A$ be an associative algebra with
  unit over $\K$, and let $\rho : \g \to A$
  be a linear map satisfying
  $\rho(x) \rho(y) - \rho(y) \rho(x) = \rho([x, y])$.
  Then $\rho$ can be extended in a unique
  way to a morphism of associative algebras,
  i.e. to a map $\rho : U(\g) \to A$.
\end{theorem}

\begin{corollary}
  Any representation of $\g$ (not necessarily
  finite-dimensional) has a canonical
  structure of a $U(\g)$-module. Conversely,
  every $U(\g)$-module has a canonical
  structure of a $\g$-representation.
\end{corollary}
