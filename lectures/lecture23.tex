\chapter{Nov.~10 --- Root Decomposition}

\section{Root Decomposition, Continued}
\begin{lemma}
  We have the following:
  \begin{enumerate}
    \item For $\alpha \in R$, we have
      $(\alpha, \alpha) = (H_\alpha, H_\alpha) \ne 0$.
    \item Let $e \in \g_\alpha$ and
      $f \in \g_{-\alpha}$
      such that $(e, f) = 2 / (\alpha, \alpha)$
      and let $h_\alpha = 2H_\alpha / (\alpha, \alpha)$.
      Then $\langle h_\alpha, \alpha \rangle = 2$
      and $e, f, h_\alpha$ form an
      $\mathfrak{sl}(2, \C)$-triple.
    \item The $h_\alpha$ defined in
      (2) does not depend on the choice of
      bilinear form.
  \end{enumerate}
\end{lemma}

\begin{proof}
  (1) Assume $(\alpha, \alpha) = 0$,
  then $(H_\alpha, \alpha) = 0$.
  Take $e \in \g_\alpha$, $f \in g_{-\alpha}$
  such that $(e, f) \ne 0$, and
  let $h = [e, f] = (e, f) H_\alpha$.
  Consider the algebra $\mathfrak{a}$ generated by
  $\{h, e, f\}$. Then
  \begin{align*}
    [h, e] &= \langle h, \alpha \rangle e = 0 \\
    [h, f] &= -\langle h, \alpha \rangle f = 0.
  \end{align*}
  Thus $\mathfrak{a}$ is solvable, so by
  Lie's theorem one can choose a basis in
  which $\ad_h, \ad_e, \ad_f$ are
  upper-triangular. Since
  $h = [e, f]$, we get that $\ad_h$ is
  strictly upper-triangular,
  so $\ad_h$ is nilpotent. But
  $h \in \mathfrak{h}$, so it is
  also semisimple, so we have
  $h = 0$.

  (2)-(3) This follows from (1) and
  Lemma \ref{lem:root-commutator}.
\end{proof}

\begin{lemma}
  Assume $\alpha$ is a root and let
  $\mathfrak{sl}(2, \C)_\alpha$ be the
  Lie subalgebra generated by
  $\{h_\alpha, e, f\}$ (where
  $e \in \g_\alpha$, $f \in \g_{-\alpha}$).
  Consider the space
  \[
    V = \C h_\alpha \oplus
    \bigoplus_{k \in \Z, k \ne 0} \g_{k\alpha} \subseteq \g.
  \]
  Then $V$ is an irreducible
  representation of $\mathfrak{sl}(2, \C)_\alpha$ (under the adjoint action).
\end{lemma}

\begin{proof}
  Note that $\ad_e \g_{k\alpha} \subseteq \g_{(k + 1)\alpha}$
  and $\ad_e \g_{-\alpha} \subseteq \C h_\alpha$, so
  $V$ is naturally a representation of
  $\mathfrak{sl}(2, \C)_\alpha$.
  Since $\langle h_\alpha, \alpha \rangle = 2 (\alpha, \alpha) / (\alpha, \alpha)$,
  we have $V[k] = 0$ for odd $k$,
  $V[2k] = \g_{k\alpha}$ for $k \ne 0$, and
  $V[0] = \C h_\alpha$. Note that
  $V[0]$ is 1-dimensional, which implies
  that $V$ is irreducible by weight
  counting.
\end{proof}

\begin{theorem}[Main theorem for the structure of semisimple Lie algebras]
  Let $\g$ be a complex semisimple
  Lie algebra and $\mathfrak{h}$
   a Cartan subalgebra. Let
   $\g = \mathfrak{h} \oplus \bigoplus_{a \in R} \g_\alpha$
   be the root decomposition
   and $(\cdot, \cdot)$ a non-degenerate
   invariant bilinear form on $\g$. Then
   \begin{enumerate}
     \item $R$ spans $\mathfrak{h}^*$
       as a vector space, and
       $\{h_\alpha\}$ span $\mathfrak{h}$ as
       a vector space.
     \item For every
       $\alpha \in R$, the root subspace
       $\g_\alpha$ is $1$-dimensional.
     \item For any roots $\alpha, \beta$,
       the number $\langle h_\alpha, \beta \rangle = 2 (\alpha, \beta) / (\alpha, \alpha)$
       is an integer.
     \item For $\alpha \in R$, define
       \emph{reflection operators}
       $s_\alpha : \mathfrak{h}^* \to \mathfrak{h}^*$ by
       \[
         s_\alpha(\lambda)
         = \lambda - \langle h_\alpha, \lambda \rangle \alpha
         = \lambda - \frac{2(\alpha, \lambda)}{(\alpha, \alpha)} \alpha.
       \]
       Then for any $\alpha, \beta \in R$,
       we have $s_\alpha(\beta) \in R$.\footnote{Note that $s_\alpha(\alpha) = -\alpha$.}
     \item For any root $\alpha$, the
       only multiples of $\alpha$ which
       are also roots are
       $0$ and $\pm \alpha$.
     \item For roots $\alpha$ and
       $\beta \ne \pm \alpha$, the
       subspace
       \[
         V = \bigoplus_{k \in \Z} g_{\beta + k\alpha}
       \]
       is an irreducible representation of
       $\mathfrak{sl}(2, \C)_\alpha$.
     \item If $\alpha, \beta \in R$ 
       such that $\alpha + \beta$ is also
       a root, then $[\g_\alpha, \g_\beta] = \g_{\alpha + \beta}$.
   \end{enumerate}
\end{theorem}

\begin{proof}
  (1) Assume $R$ does not generate all of
  $\mathfrak{*}$. Then there exists $h \in \mathfrak{h}$ 
  such that $\langle h, \alpha \rangle = 0$
  for all $\alpha \in R$ By the root
  decomposition, we have $\ad_h = 0$. But
  we have zero center since $\g$ is
  semisimple.

  (2) Let $V = \C h_\alpha \oplus \g_\alpha \oplus \g_{-\alpha}$
  be the irreduclbe representation
  of $\mathfrak{sl}(2, \C)_\alpha$, and note
  the subspaces of a given weighht are
  one-dimensional.

  (3) Let $\g$ be the representation
  of $\mathfrak{sl}(2, \C)_\alpha$. For
  elements of $\g_\beta$,
  \[
    [h_\alpha, \g_\beta]a
   = \langle h_\alpha, \beta \rangle,
   \]
   so the weight $\langle h_\alpha, \beta \rangle = 2(\alpha, \beta) / (\alpha, \alpha)$
   is an integer.

   (4) Assume
  $\langle h_\alpha, \beta \rangle = n \ge 0$.
  The elements of $\g_\beta$ have weight $n$
  with respect to
  $f^n_\alpha : V[h] \to V[-h]$.  So if
  $v \in \mathfrak{g}_\beta$, then
  $f^n_\alpha v \in \g_{\beta - n\alpha}$
  is also nonzero. Thus
  $\beta - n \alpha = s_\alpha(\beta) \in R$.
  For $n \le 0$, we can apply the
  same argument but using $e_\alpha^{-n}$
  instead of $f_\alpha^n$.

  (5) Assume $\beta = c \alpha$ for
  some $c \in \C$. Then
  $2c = 2(\alpha, \beta) / (\alpha, \alpha) \in \Z$,
  so $c \in \Z / 2$. Applying the
  same argument to $\alpha = (1 / c) \beta$,
  we get that $1 / c \in \Z / 2$ as well.
  So $c \in \{\pm 1, \pm 2, \pm 1 / 2\}$.
  By exchanging $\alpha, \beta$ and
  replacing $\alpha$ with $-\alpha$
  if necessary, we get
  $c = 1, 2$. Let
  \[
    V = \C h_\alpha \oplus
    \bigoplus_{k \in \Z, k \ne 0} \g_{k\alpha}
  \]
  be the irreducible representation of
  $\mathfrak{sl}(2, \C)_\alpha$. By
  part (2), $V[2] = g_\alpha = \C e_\alpha$,
  so $\ad_{e_\alpha} \g_\alpha = 0$.
  Thus the highest weight in $V$ is equal
  to $2$, hence
  $V = \g_{- \alpha} \oplus \C h_\alpha \oplus \g_\alpha$.
  Thus we must have $c = \pm 1$.

  (6) This follows from
  $\dim \g_{\beta + k\alpha} = 1$.

  (7) We know
  $[\g_\alpha, \g_\beta] \subseteq \g_{\alpha + \beta}$
  and $\dim \g_\alpha = 1$ for all
  $\alpha \in R$, so it remains to show
  $[e_\alpha, e_\beta] \ne 0$
  if $\alpha + \beta \in R$.
  This follows from (6) and the
  fact that if $v \in V[k]$ is nonzero and
  $V[k + 2] \ne 0$, then $e_\alpha v \ne 0$.
\end{proof}

\begin{theorem}
  We have the following:
  \begin{enumerate}
    \item Let $\mathfrak{h}_\R \subseteq \mathfrak{h}$
      be the real vector space generated
      by $h_\alpha$ for $\alpha \in R$.
      Then
      \[
        \mathfrak{h}
        = \mathfrak{h}_\R
        \oplus i \mathfrak{h}_\R
      \]
      and the restriction of the Killing
      form to $\mathfrak{h}_\R$ is
      positive-definite.
    \item Let $\mathfrak{h}_\R^* \subseteq \mathfrak{h}^*$
      be the real vector space generated by
      $\alpha \in R$. Then
      \[
        \mathfrak{h}^*
        = \mathfrak{h}_\R^*
        \oplus i \mathfrak{h}_\R^*
      \]
      and $\mathfrak{h}_\R^* = \{ \alpha \in \mathfrak{h}^* : \langle \alpha, h \rangle \in \R \text{ for all } h \in \mathfrak{h}_\R \} = (\mathfrak{h}_\R)^*$.
  \end{enumerate}
\end{theorem}

\begin{proof}
  (1) We have
  \[
    (h_\alpha, h_\beta)
    = \tr_\g(\ad_{h_\alpha} \ad_{h_\beta})
    = \sum_{\gamma \in R} \langle h_\alpha, \gamma \rangle \langle h_\beta, \gamma \rangle.
  \]
  Since $\langle h_\alpha, \gamma \rangle, \langle h_\beta, \gamma \rangle \in \Z$,
  we get that
  $(h_\alpha, h_\beta) \in \Z$. Now take
  $h = \sum c_\alpha h_\alpha \in \mathfrak{h}_\R$.
  Then
  \[
    \langle h, \gamma \rangle
    = \sum c_\alpha \langle h_\alpha, \gamma \rangle \in \R
  \]
  for all roots $\gamma$. Finally, we
  can compute that
  \[
    (h, h)
    = \tr(\ad_h \ad_h)
    = \sum_\gamma \langle h, \gamma \rangle^2 \ge 0,
  \]
  so the Killing form is positive.
  Note that it is negative definite on
  $i \mathfrak{h}_\R$, so
  $\mathfrak{h} \cap i \mathfrak{h}_\R = \{0\}$.
  Since
  \[
    \dim_\C \mathfrak{h}
    = \dim_\R \mathfrak{h}_\R
    =
    \dim_\R i \mathfrak{h}_\R
    = r
  \]
  we get the direct sum decomposition
  $\mathfrak{h} = \mathfrak{h}_\R \oplus i \mathfrak{h}_\R$.
\end{proof}

\begin{remark}
  If $\mathfrak{k}$ is a compact real form
  of $\g$ then $\mathfrak{k} \cap \mathfrak{h} = i \mathfrak{h}_\R$.
\end{remark}

\begin{example}
  In the case $\g = \mathfrak{sl}(n, \C)$,
  $\mathfrak{h}$ are the traceless
  diagonal matrices, and
  $\mathfrak{k} = \mathfrak{su}(n)$
  are the anti-Hermitian traceless
  diagonal matrices.
\end{example}

\section{Regular Elements}

\begin{example}
  Let $\g = \mathfrak{sl}(n, \C)$
  and $h \in \g$ such that the eigenvalues
  $\{\lambda_i\}$
  of $h$ are distinct (note that the
  space of such matrices is open and
  dense in $\g$). Thus
  $\ad_h$ has eigenvalues
  $\lambda_i - \lambda_j$, so
  \[
    [h, x] = 0
  \]
  if and only if $x$ is diagonal in the
  corresponding basis. Then a Cartan
  algebra is the centralizer of $h$.
\end{example}

\begin{definition}
  For any $x \in \g$, define
  the \emph{nullity} of $x$, denoted
  $n(x)$, to be the multiplicity
  of $0$ as a generalized eigenvalue of
  $\ad_x$.
\end{definition}

\begin{remark}
  For any $x \in \g$, we have
  $n(x) \ge 1$ since
  $\ad_x(x) = 0$.
\end{remark}

\begin{definition}
  For a Lie algebra $\g$,
  its \emph{rank} is
  $\rank(\g) = \min_x n(x)$. We say that an
  element $x \in \g$ is
  \emph{regular} if $n(x) = \rank(\g)$.
\end{definition}

\pagebreak
\begin{lemma}
  For any finite-dimensional complex
  Lie algebra $\g$, the set $\g^{\mathrm{reg}}$
  of regular elements in $\g$ is
  connected, open, and dense in $\g$.
\end{lemma}

\begin{prop}
  Let $\g$ be a complex semisimple
  Lie algebra and
  $\mathfrak{h}$ a Cartan subalgebra.
  Then
  \begin{enumerate}
    \item $\dim \mathfrak{h} = \rank(\g)$.
    \item $\mathfrak{h} \cap \g^{\mathrm{reg}} = \{h \in \mathfrak{h} : \langle h, \alpha \rangle \ne 0 \text{ for all } a \in R\}$
      is open and dense in
      $\mathfrak{h}$.
  \end{enumerate}
\end{prop}
